%% !TEX program = lualatex
%% !TEX encoding = UTF-8
%%
%% luatexko-doc.tex
%%
%% written by Dohyun Kim <nomos at ktug org>
%% public domain

\documentclass[a4paper]{article}
\usepackage[hmargin=3.8cm]{geometry}
\usepackage[hangul]{luatexko}
\usepackage{luamplib}
\setmainhangulfont{KoPubBatang_Pro}[
  AutoFakeSlant,
  UprightFont=* Light,
  BoldFont=* Bold,
  InterLatinCJK=.125em,
  Expansion, Protrusion,
]
\setsanshangulfont{KoPubDotum_Pro}[
  UprightFont=* Light,
  BoldFont=* Bold,
  InterLatinCJK=.125em,
]
\usepackage[svgnames]{xcolor}
\usepackage{hologo}
\def\luatex{\hologo{LuaTeX}}
\def\MP{\hologo{METAPOST}}
\def\XeTeX{\hologo{XeTeX}}
\usepackage[pdfencoding=auto,bookmarksnumbered,hidelinks]{hyperref}

\edef\verbatim{\unexpanded\expandafter{\verbatim
  \linespread{1.1}\selectfont\color{MidnightBlue}}}
\edef\itemize{\unexpanded\expandafter{\itemize\small}}
\catcode`|=13 \def|{\verb|\color{MidnightBlue}\let\\=\textbackslash} %|
\def\logoko{\textsf{k}\kern-.0625em\textit{o}}
\def\luatexko{\luatex-\logoko}
\def\texlive{\TeX\ Live}
\def\kotex{\logoko\kern-.1ex .\kern-.1667em \TeX}
\edef\section#1{\unexpanded\expandafter{\section{\empty}\vskip-5pt\hrule\nobreak\vskip5pt}}
\def\hemph#1{\textsf{\bfseries #1}}
\def\cs#1{\texttt{\color{MidnightBlue}\textbackslash\detokenize{#1}}}
\def\ldots{$\dots$}
\widowpenalty10000 \clubpenalty10000

\begin{document}
\title{\luatexko\ 간단 매뉴얼}
\author{Dohyun Kim \normalsize |<nomos at ktug org>| \and
  \normalsize \url{<http://github.com/dohyunkim/luatexko>}}
\date{Version 1.6\quad 2014/06/02}
\maketitle

\begin{quote}\small
  For a summary introduction in English, please see |README| file.

\begin{itemize}\linespread{1.1}\selectfont
    \item[v1.0] ------
    \item \hyperref[sec:loading]{플레인텍에서도 luatexko.sty을 불러야 한다.}
    \item \hyperref[sec:packageopt]{패키지 옵션 |unfonts|가 선언되지 않으면
      은글꼴을 자동으로 부르지 않는다.}
    \item \hyperref[sec:fontoption]{글꼴 옵션값을 |<dimen>|으로 지시한다.}
    \item \hyperref[sec:metapost]{\MP\ 코드를 넣으려면 luamplib 패키지를
      불러야 한다.}
    \item[v1.3] ------
    \item \hyperref[sec:fontcmds]{fallbackfont 계열의 명령이 추가되었다.}
    \item \hyperref[sec:fontswitch]{\cs{hangulpunctuations} 선언이 추가되고,
      |[QuoteRaise]| 옵션이 없어졌다.}
    \item \hyperref[sec:autojosa]{\cs{josaignoreparens=1}이 선언되면
      자동조사 결정시 괄호 부분을 무시한다.}
    \item[v1.5] ------
    \item \hyperref[sec:verttype]{세로쓰기를 지원하기 위한 실험적 코드가
      들어갔다.}
    \item \hyperref[sec:actualtext]{\cs{actualtext} 명령이 추가되었다.}
    \item \hyperref[sec:fontswitch]{한글·한자 글꼴의 지정이 없으면
      나눔 Type1을 이용한다.}
    \item \hyperref[sec:packageopt]{패키지 옵션 |[nofontspec]|이 추가되었다.}
    \item[v1.6] ------
    \item \hyperref[sec:fontoption]{글꼴 옵션 |[NoEmbedding]|이 추가되었다.}
    \item \hyperref[sec:fontcmds]{main/sans hangul/hanja/fallback 폰트의
      디폴트 옵션으로 |[Ligatures=TeX]|이 주어진다.}
  \end{itemize}
\end{quote}

\smallskip

\section{}
\hemph{\texlive\ 2013 이상}을 사용해야 한다.
세로쓰기를 위해서는 \texlive\ 2014 이상이 필요하다.
아직 \luatex은 베타 상태로 개발 중에 있으므로 안정적인 동작을 보장하지 않는다.

\section{}\label{sec:loading}
\luatexko\ 로드하기: \cs{usepackage{luatexko}}.  플레인텍은
\cs{input luatexko.sty}.  입력은 원칙적으로 UTF-8으로 한다.
BOM (Byte Order Mark)은 있어도 좋고 없어도 좋다.
CP949 인코딩에 관해서는 제\ref{sec:uhcencoding}절 참조.

\section{}\label{sec:packageopt}
패키지 옵션으로 |[hangul]|과 |[hanja]|가 제공된다.  행간격 조정이 행해지며
장 제목이나 표·그림 캡션 따위가 한글 또는 한자 방식으로 표시된다.  다만
장(chapter) 제목과 편(part) 제목에만 ``제(第)''가 붙어 ``제~1 편''
``제~3 장''과 같은 방식으로 표시되며 절 제목 이하에는 붙지 않는다.

|[unfonts]|는 은글꼴 트루타입을 불러오는 패키지 옵션이다.\footnote{%
  |nofontspec| 옵션과 함께 사용할 수 없다. 또한 이 옵션은
  deprecated 상태로서 2015년에는 사라질 예정이다.}
이전 버전과 달리 은글꼴이 기본으로 로드되지 않음에 유의할 것.

|[nofontspec]| 옵션을 주면 fontspec을 부르지 않는다.
따라서 아래 \ref{sec:fontcmds}, \ref{sec:fontswitch}, \ref{sec:fontoption},
\ref{sec:verttype}, \ref{sec:mathhangul}절의 유니코드 한국어 글꼴 설정
명령들도 사용할 수 없다.

\section{}\label{sec:fontcmds}
\luatexko를 로드하면 fontspec 패키지를 자동으로 불러온다.  글꼴 설정에
대해서는 fontspec 문서를 참조하라.

한국어 글꼴을 위해 새로 마련한 명령은 다음과 같다.
\begin{verbatim}
  \setmainhangulfont     \setmainhanjafont    \setmainfallbackfont
  \setsanshangulfont     \setsanshanjafont    \setsansfallbackfont
  \setmonohangulfont     \setmonohanjafont    \setmonofallbackfont
  \newhangulfontfamily   \newhanjafontfamily  \newfallbackfontfamily
  \addhangulfontfeature  \addhanjafontfeature \addfallbackfontfeature
  \hangulfontspec        \hanjafontspec       \fallbackfontspec
\end{verbatim}
\cs{adhochangulfont} \cs{adhochanjafont} \cs{addhocfallbackfont}는 각각
\cs{hangulfontspec} \cs{hanjafontspec} \cs{fallbackfontspec}의 다른 이름이다.
main/sans hangul/hanja/fallback 글꼴에는 |Ligatures=TeX| 옵션이 자동으로
주어진다.
\begin{itemize}\item[]
플레인텍에서 한글 글꼴 설정은 영문 글꼴 지정하는 방식과 거의 같다.
\begin{verbatim}
  \hangulfont=UnDotum\relax
  \hanjafont=UnDotum at 14pt
  \fallbackfont=HanaMinB at 12pt
  \hangulfont=UnDotum scaled 2000
  \hanjafont="HCR Batang LVT"\relax
  \hangulfont={HCR Batang LVT:script=hang;+dlig} at 12pt
\end{verbatim}
\end{itemize}

\section{}\label{sec:fontswitch}
원칙적으로 \luatexko는 지시된 글자가 영문폰트에 없으면 한글폰트에서,
한글폰트에도 없으면 한자폰트에서, 한자폰트에도 없으면 fallback
폰트에서 글자를 찾아 찍는다. 세 가지 \hemph{모두 지정되지 않았다면 나눔 Type1
폰트를} 이용한다. 기존 \kotex과는 글꼴 대체 방식이 다르므로 주의해야 한다.

하지만 \cs{hanjabyhanjafont=1}을 선언하면 한자는 우선 한자글꼴로
식자한다.  |0|을 선언하면 원래 방식으로 되돌아간다. 이 명령은 문단
중간에서 사용하지 않도록 주의해야 한다. 문단 중에 사용되었다면
그 문단의 처음부터 효력이 발생한다.

\cs{hangulpunctuations=1}을 선언하면 영문 \hemph{문장부호들을 한글 폰트로}
식자한다. |0|을 지시하면 원래 상태로 되돌린다.
|[hangul]| 옵션을 주면 \cs{hangulpunctuations=1}이 자동으로
선언된다.\footnote{나눔 Type1 상태면 그러하지 아니하다.}
이 선언에 의해 영향 받는 문장부호들을 다음처럼 지정할 수 있다.
인자는 콤마로 연결된 숫자 형식으로서 유니코드 코드포인트를 뜻한다.
\begin{verbatim}
  \registerpunctuations{45, "2D, `-}
  \unregisterpunctuations{"2D, "2015}
\end{verbatim}


\section{}\label{sec:fontoption}
fontspec의 글꼴 옵션 외에 \luatexko가 따로 제공하는 것들이
있다.\footnote{ 옛한글 식자를 위해서는
  |[Script=Hangul]| 옵션을 사용하는 것으로 충분하다.}

그러나 \cs{defaulthangulfontfeatures} 따위 명령은 따로 구현돼 있지 않으며,
설혹 사용하더라도 fontspec의 \cs{defaultfontfeatures}와 같은 효과를 가진다.
다시 말해 한글 폰트 옵션과 라틴 폰트 옵션에 차이는 없다.

\begin{figure}
  \fboxsep=1em
  \fbox{\parbox{\dimexpr\linewidth-2.1em\relax}{%
      \fontspec{HCR Dotum LVT}[Script=Hangul,Ligatures=Required]
      \leftskip2.5cm
뎨 가ᄂᆞᆫ 뎌 각시 본 듯도 ᄒᆞᆫ뎌이고.\\
天텬上상白ᄇᆡᆨ玉옥京경을 엇디ᄒᆞ야 離니別별ᄒᆞ고\\
ᄒᆡ 다 져믄 날의 눌을 보라 가시ᄂᆞᆫ고

\medskip

어와 네여이고 내 ᄉᆞ셜 드러보오.\\
내 얼굴 이 거동이 님 괴얌즉 ᄒᆞᆫ가마ᄂᆞᆫ\\
엇딘디 날 보시고 네로다 녀기실ᄉᆡ\\
나도 님을 미더 군 ᄠᅥ디 전혀 업서\\
이ᄅᆡ야 교ᄐᆡ야 어ᄌᆞ러이 구돗ᄯᅥᆫ디\\
반기시ᄂᆞᆫ ᄂᆞᆺ비치 녜와 엇디 다ᄅᆞ신고.%\\
%누어 ᄉᆡᆼ각ᄒᆞ고 니러 안자 혜여ᄒᆞ니\\
%내 몸의 지은 죄 뫼ᄀᆞ티 ᄡᅡ혀시니\\
%하ᄂᆞᆯ히라 원망ᄒᆞ고 사ᄅᆞᆷ이라 허믈ᄒᆞ랴\\
%셜워 플텨 혜니 造조物믈의 타시로다.
}}
\caption{옛한글 조판 보기}\label{fig:yethangul}
\end{figure}

\begin{description}
  \item[InterHangul] {\addhangulfontfeature{InterHangul=.5ex}한글 글자
    사이의 자간}.  아래는 $-0.04$em 만큼 한글
  자간을 지시한다. 플레인텍에서는 |interhangul|.
\begin{verbatim}
  [InterHangul=-0.04em]
\end{verbatim}

\item[InterLatinCJK] 한글 또는 한자와 라틴 문자 사이의 자간을
  설정한다. 플레인텍에서는 |interlatincjk|.
\begin{verbatim}
  [InterLatinCJK=0.125em]
\end{verbatim}

\item[CharRaise] 글자의 세로 위치를
  {\addhangulfontfeature{CharRaise=.3em} 조절}할 수 있는 기능이다.
  이로써 주변에 식자되는 다른 글꼴과 조화를 이루게 한다.
  플레인텍에서는 |charraise|.
\begin{verbatim}
  [CharRaise=0.1em]
\end{verbatim}

\item[PunctRaise] 한글 또는 한자 다음에 라틴 구두점---마침표에
  한한다---이 왔을 때 그 세로 위치를 조절할 수 있다.
  \cs{hangulpunctuations}로 인해 거의 쓸 일이 없게 되었다.
  플레인텍에서는 |punctraise|.
\begin{verbatim}
  [PunctRaise=-0.125em]
\end{verbatim}

\item[NoEmbedding] |NoEmbed|와 동의어.
  폰트를 내장하지 않으므로 결과물의 크기가 대폭 줄어든다.
  하지만 대외적 배포에 적합하지 않음에 유의.\footnote{%
    Adobe Reader에 번들된 폰트를 쓴다면 대외적 배포도 대체로 무난하다.}
  플레인텍에서는 |embedding=no|.
\begin{verbatim}
  [NoEmbedding]
\end{verbatim}

\item[Protrusion] 특정 글자가 행 끌에 왔을 때 판면 바깥으로 끌어내는
  기능이다.  Lua\LaTeX의 기본값은 구두점들만 완전히 글자내밀기 한다. 즉
  hanging punctuation이 작동한다.
\begin{verbatim}
  [Protrusion]
\end{verbatim}
  이는 |Protrusion=default|와 같은 뜻이다.
  플레인텍에서는 |protrusion=default|.
  마이크로타입에 관심 있으면 자신만의 설정을 만들어 지정할 수 있다.

\item[Expansion] 판면의 균일한 조판을 위해 글자들을 미세하게 늘이거나
  줄이는 기능이다.
\begin{verbatim}
  [Expansion]
\end{verbatim}
  이는 |Expansion=default|와 마찬가지 뜻이다.
  플레인텍에서는 |expansion=default|.

\end{description}

\section{}
고문헌·일본어·중국어 조판을 위해 japanese, Schinese, Tchinese 환경을
제공한다.  chinese는 Schinese의 다른 이름이다.  korean 환경도
마련했는데 이들 환경 안에서 잠깐 한국어를 조판하는 데
사용한다.  일본어·중국어라도 글꼴 설정은 \cs{newhangulfontfamily}
\cs{newhanjafontfamily} 따위를 이용한다.  그림~\ref{fig:ancientdoc} 참조.
플레인텍에서는 {\small(문서 전체에 적용하지 않는다면 그룹을 열고)}
\cs{chinese} \cs{japanese} 따위를 사용한다.

\begin{figure}
  \centering
  \fbox{\parbox{37em}{\begin{chinese}\setmainfont{STFangsong}
子曰:「學而時習之,不亦說乎?有朋自遠方來,不亦樂乎?人不知而不慍,
不亦君子乎?」\par
有子曰:「其爲人也孝弟,而好犯上者,鮮矣!不好犯上,而好作亂者,未之有也!
君子務本,本立而道生;孝弟也者,其爲仁之本與?」\par
子曰:「巧言令色,鮮矣仁!」\par
曾子曰:「吾日三省吾身:爲人謀,而不忠乎?與朋友交,而不信乎?傳,
不習乎?」\par
子曰:「道千乘之國,敬事而信,節用而愛人,使民以時。」\par
子曰:「弟子入則孝,出則弟;謹而信,汎愛衆;而親仁,行有餘力,則以學文。」
  \end{chinese}}}
\caption{고문헌 조판 보기}\label{fig:ancientdoc}
\end{figure}

\luatexko가 글자 사이에 삽입하는 미세간격을 사용자가 영(zero)으로
강제하기 위해선 \cs{inhibitglue} 명령을 이용한다.
대체로 일본어·중국어 환경에서만 문제된다.

\section{}\label{sec:verttype}
세로쓰기는 폰트의 고급 오픈타입 자질을 이용하므로 폰트가 이를 지원해야
가능한 일이다. 폰트에 |Vertical=RotatedGlyphs| 옵션을 준다.
플레인텍이라면 |vertical;+vrt2| 옵션. |vrt2| 자질을 갖지 않는 폰트를 쓰는
경우 |vert| 자질을 명시해주어야 한다. 둘 다 없으면 세로쓰기에 적합하지 않은
글꼴이다.
세로쓰기에는 \hemph{\luatex\ 0.79 이상}, 즉 \texlive\ 2014 이후
버전이 필요하다.

문서의 일부를 세로쓰기하려면 \cs{begin{vertical}{<dimen>}} \ldots\ \cs{end{vertical}}
환경을 이용하라. |<dimen>|으로 세로쓰기 박스의 높이를 지시한다.
그림~\ref{fig:vertical} 참조.
플레인텍에서는 \cs{vertical{<dimen>}} \ldots\ \cs{endvertical}.

\begin{figure}
\framebox[\linewidth]{\begin{vertical}{17em}
\fontspec{Adobe Myungjo Std}[Vertical=RotatedGlyphs,RawFeature=+vhal]
\parindent-1em\leftskip1em \linespread{1.5}\selectfont
\noindent 님의 침묵(The Silent Beloved)
\smallbreak
\hfil\hfil 한 용 운\hfil
\bigbreak
님은 갓슴니다 아아 사랑하는나의님은 갓슴니다\par
푸른산빗을깨치고 단풍나무숩을향하야난 적은길을 거러서 참어떨치고
갓슴니다\par
黃金의꽃가티 굿고빗나든 옛盟誓는 차듸찬띠끌이되야서 한숨의 微風에
나러갓슴니다\par
날카로은 첫〈키쓰〉의追憶은 나의運命의指針을 돌너노코 뒷거름처서 사러젓슴니다\par
%…… \par
\hellipsis\par
아아 님은갓지마는 나는 님을보내지 아니하얏슴니다\par
제곡조를못이기는 사랑의노래는 님의沈默을 휩싸고돔니다\par
\end{vertical}}
\caption{세로쓰기의 예}\label{fig:vertical}
\end{figure}

문서 전체를 세로쓰기한다면 이 환경을 쓰는 대신
\cs{verticaltypesetting} 명령을 전처리부에 선언한다.
이때 면주의 폰트에는 세로쓰기 옵션이 없어야 할 것이다.

\section{}\label{sec:metapost}
전처리부에서 \cs{usepackage{luamplib}}을 선언하면 \MP\ 코드를 문서 중간에
삽입할 수 있다.  한글이나 수식은 |btex| \ldots\ |etex| 안에 넣어야 한다.
그림~\ref{fig:mplib} 참조. 상세한 것은 luamplib 패키지 문서를 참조하라.

\begin{figure}
\setbox0\vbox{\kern10pt
\begin{verbatim}
  \usepackage{luamplib}
  ...
  \begin{mplibcode}
    beginfig(1);
      draw fullcircle scaled 2cm;
      dotlabel.bot(btex \TeX etex, origin);
      dotlabel.rt(btex 루아 etex, dir45*1cm);
    endfig;
  \end{mplibcode}
\end{verbatim}}
\begin{mplibcode}
beginfig(1);
  draw btex \copy0 etex shifted (-\mpdim{\textwidth-3cm},-\mpdim{.5\ht0});
  draw fullcircle scaled 2cm;
  dotlabel.bot(btex\TeX etex, origin);
  dotlabel.rt(btex 루아 etex, dir 45*1cm);
  bboxmargin:=0; draw bbox currentpicture;
endfig;
\end{mplibcode}
\caption{mplib 용례}\label{fig:mplib}
\end{figure}

\section{}\label{sec:mathhangul}
\begin{quote}
  |$가^{나^다}$|\quad$\Rightarrow\quad가^{나^다}$
\end{quote}
수식 모드에서도 한글을 {\small(hbox로 감싸지 않고)} 직접 입력할 수
있다.  문서의 기본 한글 글꼴이 자동으로 수식 한글에도 적용되므로 따로
설정할 것이 없지만 굳이 한다면 다음처럼 지시한다.
\begin{verbatim}
  \setmathhangulfont{HCRBatang}
\end{verbatim}
현재 한글만 쓸 수 있게 설정되어 있다.
한자도 수식에 직접 입력하려면 사용자는
\begin{verbatim}
  \setmathhangulblock{4E00}{9FC3}
\end{verbatim}
명령으로 유니코드 블럭을 추가 지정해야 한다.

\section{}
\cs{dotemph} 명령으로 \dotemph{드러냄표}%
를 이용한 강조를 할 수 있다.  기본은 글자 위에 점을 찍는 형태이나
다음과 같이 명령을 주어 개인적으로 선호하는 기호를 드러냄표로 쓸 수
있다.

①~|\def\dotemphraise{0.4em }|: 드러냄표를 피강조 글자 위로 끌어올리는 길이

②~|\def\dotemphchar{\bfseries ^^^^02d9}|: 드러냄표 기호 자체를 정의.
|^^^^02d9|는 유니코드 코드포인트를 뜻하는 16진수이고 소문자로만 써야 한다.
숫자 대신 직접 문자를 입력해도 된다. 플레인텍에서도 쓸 수 있다.

\section{}
루비를 달 수 있다. ruby 패키지가 이미 존재하지만 \luatexko와 궁합이 잘
맞지 않아 새로 매크로를 제공한다.  플레인텍도 지원한다.
\begin{quote}
  \cs{ruby{漢字}{한자}}\quad$\Rightarrow$\quad\ruby{漢字}{한자}
\end{quote}
이처럼 글자별로 따로 루비를 달 필요가 없다.  관련 설정은 다음처럼
한다.

①~\cs{rubyfont}: 루비를 식자할 폰트를 지시해 둔다. 기본값은 현재 폰트.

②~|\def\rubysize{0.6}|: 루비 글자 크기를 본문 글자 크기에 대한 비율로
지정

③~|\def\rubysep{0.2ex}|: 루비와 본문 글자 사이의 간격을 지정

④~\cs{rubynooverlap}: 루비의 폭이 본문 글자의 폭보다 클 때 루비가 이웃
글자들 위로 삐져나가지 못하게 한다. 본문 글자의 흐름을 중시하여
\cs{rubyoverlap}을 기본값으로 하였으므로 이는 따로 선언할 필요가 없다.

\section{}
ulem 패키지가 \luatexko와 궁합이 잘 맞지 않아{\small (줄바꿈에 문제가 있음)}
명령을 따로 제공한다. 플레인텍에서도 쓸 수 있다.

\bigskip
\halign{\qquad#\hfil&\quad$\Rightarrow$\quad#\hfil\cr
\cs{uline{밑줄을 그을 수 있다}}&\uline{밑줄을 그을 수 있다}\cr
\cs{sout{취소선을 그을 수 있다}}&\sout{취소선을 그을 수 있다}\cr
\cs{uuline{밑줄을 두 줄 긋는다}}&\uuline{밑줄을 두 줄 긋는다}\cr
\cs{xout{빗금으로 취소할 수 있다}}&\xout{빗금으로 취소할 수 있다}\cr
\cs{uwave{물결표로 밑줄을 삼는다}}&\uwave{물결표로 밑줄을 삼는다}\cr
\cs{dashuline{대시로 밑줄을 삼는다}}&\dashuline{대시로 밑줄을 삼는다}\cr
\cs{dotuline{밑줄을 점선으로 긋는다}}&\dotuline{밑줄을 점선으로 긋는다}\cr
}
\bigskip

관련하여 다음 설정을 할 수 있다.

①~|\def\ulinedown{0.25em}|: 밑줄을 베이스라인 아래로 끌어내리는 정도

②~|\def\ulinewidth{0.04em}|: 밑줄의 굵기

\section{}\label{sec:autojosa}
자동조사는 \kotex 과 동일하게 \cs{은} \cs{는} \cs{이} \cs{가} \cs{을} \cs{를}
\cs{와} \cs{과} \cs{로} \cs{으로} \cs{라} \cs{이라} 따위를 사용한다.
문장 중에서도 작동할 뿐만 아니라 플레인텍도 지원한다.
버전 1.3부터는 \cs{josaignoreparens=1}이 선언되어 있으면 자동조사는
\hemph{괄호 부분을 건너뛰고} 그 앞 글자에 매칭한다.
|0|이 선언되면 원래 방식으로 돌아간다.
\begin{quote}
  \cs{josaignoreparens=1} \josaignoreparens=1 \\
  |홍길동(1992)\는|\quad$\Rightarrow$\quad 홍길동(1992)\는\\
  |홍길동(2001)\로|\quad$\Rightarrow$\quad 홍길동(2001)\로\par
  \cs{josaignoreparens=0} \josaignoreparens=0 \\
  |홍길동(1992)\는|\quad$\Rightarrow$\quad 홍길동(1992)\는\\
  |홍길동(2001)\로|\quad$\Rightarrow$\quad 홍길동(2001)\로
\end{quote}

\section{}
항목 번호를 한국어 기호로 붙일 수 있다. \kotex과 동일하게 \cs{jaso} \cs{gana}
\cs{ojaso} \cs{ogana} \cs{pjaso} \cs{pgana} \cs{onum} \cs{pnum} \cs{oeng}
\cs{peng} \cs{hnum} \cs{Hnum} \cs{hroman} \cs{hRoman} \cs{hNum} \cs{hanjanum}
따위를 사용한다.

\section{}\label{sec:actualtext}
\cs{actualtext{...}} 명령은 인자를 식자함과 동시에, \hemph{입력한 문자 그대로}
PDF에서 텍스트로 추출할 수 있게 해준다. 인자가 두 페이지에 나눠지지 않도록
유의한다.  모든 PDF 리더가 이를 지원하는 것은 아니다. 예:
$\actualtext{\sqrt 2}$,
{\fontspec{HCR Dotum LVT}[Script=Hangul,Ligatures=Required]
\actualtext{ᄆᆞᄎᆞᆷ〮내〯}}.
인자가 글자 없이 그림으로만 돼있다면 \cs{actualtext*{...}} 방식을 이용한다.

\section{}
\cs{luatexhangulnormalize=1}이라 지시하면 첫가끝 자모를 완성형 음절로,
|2|라면 완성형 음절을 첫가끝 자모로 인코딩 변환한다. |0|이 할당되면
인코딩 변환 기능이 꺼진다. \XeTeX의 \cs{XeTeXinputnormalization} 명령과
유사하나 오직 한글과 일부 한자에 대해서만 정규화가 작동하는 점에서
\XeTeX의 그것에 비해 기능이 한참 모자란다.

\section{}\label{sec:uhcencoding}
권장하지 않지만 불가피하게 입력 인코딩이 UHC (Unified Hangul Code)\footnote{%
  CP949라고도 하며 EUC-KR을 포함한다}로 되어 있는 파일을 처리할 때는
\cs{luatexuhcinputencoding=1}을 선언한다.
|0|을 할당하면 다시 UTF-8 입력으로 간주한다.
\XeTeX의 \cs{XeTeXinputencoding} 명령과 유사하나 오직 한국어 문자만 처리할 수
있어 \XeTeX의 그것에 비해 기능이 한참 모자란다.

\section{}
마찬가지로 바람직하지는 않지만 불가피하게 파일 이름이 UHC로 인코딩되어
있다면 \cs{luatexuhcfilenames=1}을 선언한다. |0|을 할당하면 다시 UTF-8
이름으로 간주한다.  윈도 계열 운영체제에서만 문제될 것이다.
\hfill \fboxsep=-\fboxrule \fbox{\vbox to 1em{\hbox to 1em{\hss}\vss}}

\end{document}
