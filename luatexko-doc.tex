%% !TEX program = lualatex
%% !TEX encoding = UTF-8
%%
%% luatexko-doc.tex
%%
%% written by Dohyun Kim <nomos at ktug org>
%% public domain

\documentclass[a4paper]{article}
\usepackage{typearea}
\usepackage[hangul]{luatexko}
\usepackage{luamplib}
\setmainhangulfont[
  InterLatinCJK=.125em,
  QuoteRaise=.1em,
  Expansion,
  ]{NanumMyeongjo}
\setmainhanjafont{UnBatang}
\setsanshangulfont[
  InterLatinCJK=.125em,
  ]{NanumGothic}
\usepackage[svgnames]{xcolor}
\usepackage{mflogo}
\usepackage{metalogo}
\usepackage[unicode,bookmarksnumbered]{hyperref}

\edef\verbatim{\unexpanded\expandafter{\verbatim
  \linespread{1.1}\selectfont\color{MidnightBlue}}}
\edef\itemize{\unexpanded\expandafter{\itemize\small}}
\catcode`|=13 \def|{\verb|\color{MidnightBlue}} %|
\let\luatex\LuaTeX
\def\luatexko{\luatex-k$o$}
\def\texlive{\TeX\ Live}
\def\kotex{\textsf{k}\kern-.1em \textit{o}\kern-.1ex .\kern-.1667em \TeX}
\edef\section#1{\unexpanded\expandafter{\section{\empty}\vskip-5pt\hrule\nobreak\vskip5pt}}
\def\hemph#1{\textsf{\bfseries #1}}

\begin{document}
\title{\luatexko\ 간단 매뉴얼}
\author{Dohyun Kim \normalsize <|nomos at ktug org|> \and
  \normalsize <|http://github.com/dohyunkim/luatexko|>}
\date{Version 1.1\quad\today}
\maketitle

\begin{quote}\small
  For summary introduction in English, please see |README| file in this package.

  주요 변경 사항:
  \begin{itemize}\itemsep=0pt
    \item 플레인텍에서도 luatexko.sty을 불러야 한다.
    \item 패키지 옵션 |unfonts|가 선언되지 않으면 은글꼴을 자동으로
      부르지 않는다.
    \item 글꼴 옵션에 |QuoteRaise|가 추가되었다. 또한 옵션값을
      |<dimen>|으로 지시한다.
    \item 세로쓰기를 더 이상 공식적으로 지원하지 않는다.
    \item \MP\ 코드를 넣으려면 luamplib 패키지를 불러야 한다.
  \end{itemize}
\end{quote}

\section{}
\hemph{\texlive\ 2013 이상}을 사용해야 한다.
아직 \luatex은 베타 상태로 개발 중에 있으므로 안정적인 동작을 보장하지 않는다.

\section{}
\luatexko\ 로드하기: |\usepackage{luatexko}|.  플레인텍은
|\input luatexko.sty|.  입력은 원칙적으로 UTF-8으로 한다.
BOM (Byte Order Mark)은 있어도 좋고 없어도 좋다.
CP949 인코딩에 관해서는 제\ref{sec:uhcencoding}절 참조.

\section{}
패키지 옵션으로 |[hangul]|과 |[hanja]|가 제공된다.  장 제목이나
표·그림 캡션 따위가 한글 또는 한자 방식으로 표시된다.  다만
장(chapter) 제목과 편(part) 제목에만 ``제(第)''가 붙어 ``제~1 편''
``제~3 장''과 같은 방식으로 표시되며 절 제목 이하에는 붙지 않는다.

또한 |[unfonts]|는 은글꼴을 불러오는 패키지 옵션이다.
이전 버전과 달리 \hemph{은글꼴이 기본으로 로드되지 않음}에 유의할 것.

\section{}
\luatexko를 로드하면 fontspec 패키지를 자동으로 불러온다.  글꼴 설정에
대해서는 fontspec 문서를 참조하라.

한국어 글꼴을 위해 새로 마련한 명령은 다음과 같다.
\begin{verbatim}
  \setmainhangulfont     \setmainhanjafont
  \setsanshangulfont     \setsanshanjafont
  \setmonohangulfont     \setmonohanjafont
  \newhangulfontfamily   \newhanjafontfamily
  \addhangulfontfeature  \addhanjafontfeature
  \hangulfontspec        \hanjafontspec
\end{verbatim}
|\adhochangulfont| |\adhochanjafont|는 각각 |\hangulfontspec|
|\hanjafontspec|의 다른 이름이다.
\begin{itemize}\item[]
플레인텍에서 한글 글꼴 설정은 영문 글꼴 지정하는 방식과 거의 같다.
\begin{verbatim}
  \hangulfont=UnDotum\relax
  \hanjafont=UnDotum at 14pt
  \hangulfont=UnDotum scaled 2000
  \hanjafont{HCR Batang LVT}\relax
  \hangulfont={HCR Batang LVT:script=hang;+dlig} at 12pt
\end{verbatim}
\end{itemize}

\section{}
원칙적으로 \luatexko에서는 영문폰트에 지정한 글자가 없으면 한글폰트에서,
한글폰트에도 없으면 한자폰트에서 글자를 찾아 찍는다.  기존 \kotex과는
글꼴 대체 방식이 다르므로 주의해야 한다.

하지만 |\hanjabyhanjafont=1|을 선언하면 한자는 우선 한자글꼴로
식자한다.  |0|을 선언하면 원래 방식으로 되돌아간다. 이 명령은 문단
중간에서 사용하지 않도록 주의해야 한다. 문단 중에 사용되었다면
그 문단의 처음부터 효력이 발생한다.

\section{}
fontspec의 글꼴 옵션 외에 \luatexko가 따로 제공하는 것들이
있다.\footnote{ 옛한글 식자를 위해서는
  |[Script=Hangul]| 옵션을 사용하는 것으로 충분하다. 그러나
  |[YetHangul]| 옵션을 써도 무해하다.}

\begin{figure}
  \fboxsep=1em
  \fbox{\vbox{\hsize\dimexpr\linewidth-2.1em\relax
%      \fontspec[Script=Hangul]{malgungothic}
      \fontspec[Script=Hangul]{HCR Dotum LVT}
      \centering
뎨 가ᄂᆞᆫ 뎌 각시 본 듯도 ᄒᆞᆫ뎌이고.\\
天텬上상白ᄇᆡᆨ玉옥京경을 엇디ᄒᆞ야 離니別별ᄒᆞ고\\
ᄒᆡ 다 져믄 날의 눌을 보라 가시ᄂᆞᆫ고

\medskip

어와 네여이고 내 ᄉᆞ셜 드러보오.\\
내 얼굴 이 거동이 님 괴얌즉 ᄒᆞᆫ가마ᄂᆞᆫ\\
엇딘디 날 보시고 네로다 녀기실ᄉᆡ\\
나도 님을 미더 군 ᄠᅥ디 전혀 업서\\
이ᄅᆡ야 교ᄐᆡ야 어ᄌᆞ러이 구돗ᄯᅥᆫ디\\
반기시ᄂᆞᆫ ᄂᆞᆺ비치 녜와 엇디 다ᄅᆞ신고.\\
%누어 ᄉᆡᆼ각ᄒᆞ고 니러 안자 혜여ᄒᆞ니\\
%내 몸의 지은 죄 뫼ᄀᆞ티 ᄡᅡ혀시니\\
%하ᄂᆞᆯ히라 원망ᄒᆞ고 사ᄅᆞᆷ이라 허믈ᄒᆞ랴\\
%셜워 플텨 혜니 造조物믈의 타시로다.
}}
\caption{옛한글 조판 보기}\label{fig:yethangul}
\end{figure}

\begin{description}
\item[InterHangul] 한글 글자 사이의 자간.  아래는 $-0.04$em 만큼 한글
  자간을 지시한다.
\begin{verbatim}
  [InterHangul=-0.04em]
\end{verbatim}

\item[InterLatinCJK] 한글 또는 한자와 라틴 문자 사이의 자간을
  설정한다.
\begin{verbatim}
  [InterLatinCJK=0.125em]
\end{verbatim}

\item[PunctRaise] 한글 또는 한자 다음에 라틴 구두점이 왔을 때 그 세로
  위치를 조절할 수 있다.
\begin{verbatim}
  [PunctRaise=-0.125em]
\end{verbatim}

\item[QuoteRaise] 한글 또는 한자 주변에 라틴 인용부호나 \hemph{괄호}가 왔을 때
  그 세로 위치를 조절할 수 있다.
\begin{verbatim}
  [QuoteRaise=0.125em]
\end{verbatim}

\item[CharRaise] 글자의 세로 위치를
  {\addhangulfontfeature{CharRaise=.3em} 조절}할 수 있는 기능이다.
  이로써 주변에 식자되는 다른 글꼴과 조화를 이루게 한다.
\begin{verbatim}
  [CharRaise=0.1em]
\end{verbatim}

\item[Protrusion] 특정 글자가 행 끌에 왔을 때 판면 바깥으로 끌어내는
  기능이다.  Lua\LaTeX의 기본값은 구두점들만 완전히 글자내밀기 한다. 즉
  hanging punctuation이 작동한다.
\begin{verbatim}
  [Protrusion]
\end{verbatim}
  이는 |Protrusion=default|와 마찬가지 뜻이다.  마이크로타입에 관심
  있으면 자신만의 설정을 만들어 지정할 수 있다.

\item[Expansion] 판면의 균일한 조판을 위해 글자들을 미세하게 늘이거나
  줄이는 기능이다.
\begin{verbatim}
  [Expansion]
\end{verbatim}
  이는 |Expansion=default|와 마찬가지 뜻이다.

\end{description}

\section{}
\begin{figure}\linespread{1.388}\selectfont\hfil
\fbox{\vbox{\hsize=36em \begin{chinese} \setmainfont{STFangsong}
子曰:「學而時習之,不亦說乎?有朋自遠方來,不亦樂乎?人不知而不慍,不亦君子乎?」

有子曰:「其為人也孝弟,而好犯上者,鮮矣!不好犯上,而好作亂者,未之有也!君子務本,本立而道生;孝弟也者,其為仁之本歟?」

子曰:「巧言令色,鮮矣仁!」

曾子曰:「吾日三省吾身:為人謀,而不忠乎?與朋友交,而不信乎?傳,不習乎?」

子曰:「道千乘之國,敬事而信,節用而愛人,使民以時。」
\end{chinese}}}\hfil
\caption{고문헌 조판 보기}\label{fig:ancientdoc}
\end{figure}

고문헌·일본어·중국어 조판을 위해 japanese, Schinese, Tchinese 환경을
제공한다.  chinese는 Schinese의 다른 이름이다.  korean 환경도
마련했는데 이들 환경 안에서 잠깐 한국어를 조판하는 데
사용한다.  일본어·중국어라도 글꼴 설정은 |\newhangulfontfamily|
|\newhanjafontfamily| 따위를 이용한다.  그림~\ref{fig:ancientdoc} 참조.

\section{}
\luatexko가 글자 사이에 삽입하는 미세간격을 사용자가 영(zero)으로
강제하기 위해선 |\inhibitglue| 명령을 이용한다.
대체로 일본어·중국어 환경에서만 문제된다.

\section{}
\begin{quote}
  |$가^{나^다}$|\quad$\Rightarrow\quad가^{나^다}$
\end{quote}
수식 모드에서도 한글을 {\small(hbox로 감싸지 않고)} 직접 입력할 수
있다.  문서의 기본 한글 글꼴이 자동으로 수식 한글에도 적용되므로 따로
설정할 것이 없지만 굳이 한다면 다음처럼 지시한다.
\begin{verbatim}
  \setmathhangulfont{HCRBatang}
\end{verbatim}
현재 한글만 쓸 수 있게 설정되어 있다.

\section{}
\luatex 에서는 아직 \hemph{세로쓰기를 지원하지 않는다}.%
\footnote{굳이 세로쓰기를 실험해보고자 한다면 |ttx -t vmtx <폰트>.ttf|
  명령의 결과물인 |<폰트>.ttx| 파일을 현재 디렉토리에 두어야 한다. 그런 다음
  |\string\begin\string{vertical\string}\string{<dimen>\string}
  ...\string\end\string{vertical\string}| 환경을 이용하라. ttx에 대해선
  \url{http://sourceforge.net/projects/fonttools/} 참조.
  문서 전체에 대해 세로쓰기를 하려면 이 환경을 쓰는 대신
  |\string\verticaltypesetting| 명령을 전처리부에 선언한다.}

%\begin{figure}
%\hfil\framebox[\linewidth]{\begin{vertical}{16em}
%\fontspec[Vertical=RotatedGlyphs]{Adobe Myungjo Std Medium}
%\parindent-1em\leftskip1em \linespread{1.5}\selectfont
%\noindent 님의 침묵{\small(The Silent Beloved)}
%\smallbreak
%\hfil\hfil 한 용 운\hfil
%\bigbreak
%님은 갓슴니다 아아 사랑하는나의님은 갓슴니다\par
%푸른산빗을깨치고 단풍나무숩을향하야난 적은길을 거러서 참어떨치고
%갓슴니다\par
%黃金의꽃가티 굿고빗나든 옛盟誓는 차듸찬띠끌이되야서 한숨의 微風에
%나러갓슴니다\par
%날카로은 첫〈키쓰〉의追憶은 나의運命의指針을 돌너노코 뒷거름처서 사러젓슴니다\\
%%…… \par
%\hellipsis\par
%아아 님은갓지마는 나는 님을보내지 아니하얏슴니다\par
%제곡조를못이기는 사랑의노래는 님의沈默을 휩싸고돔니다\par
%\end{vertical}}
%\caption{세로쓰기의 예}\label{fig:vertical}
%\end{figure}

\section{}
전처리부에서 |\usepackage{luamplib}|을 선언하면 \MP\ 코드를 문서 중간에 삽입할 수 있다.
한글이나 수식은 |btex| ... |etex| 안에 넣어야 한다. 그림~\ref{fig:mplib} 참조.

\begin{figure}
\setbox0\hbox{%
\begin{mplibcode}
beginfig(1);
    draw fullcircle scaled 2cm;
    dotlabel.bot(btex\TeX etex, origin);
    dotlabel.rt(btex 루아 etex, dir 45*1cm);
endfig;
\end{mplibcode}
}%
\setbox2\vbox{\bigskip
\begin{verbatim}
  \usepackage{luamplib}
  ...
  \begin{mplibcode}
    beginfig(1);
      draw fullcircle scaled 2cm;
      dotlabel.bot(btex \TeX etex, origin);
      dotlabel.rt(btex 루아 etex, dir 45*1cm);
    endfig;
  \end{mplibcode}
\end{verbatim}}\wd2=.7\linewidth
\framebox[\linewidth]{\box2 \raise.8cm\box0 }
\caption{mplib 용례}\label{fig:mplib}
\end{figure}

\section{}
|\dotemph| 명령으로 \dotemph{드러냄표}%
를 이용한 강조를 할 수 있다.  기본은 글자 위에 점을 찍는 형태이나
다음과 같이 명령을 주어 개인적으로 선호하는 기호를 드러냄표로 쓸 수
있다.
\begin{verbatim}
  \def\dotemphraise{0.4em }
  \def\dotemphchar{\bfseries ^^^^02d9}
\end{verbatim}
전자는 드러냄표를 피강조 글자 위로 끌어올리는 길이를, 후자는 드러냄표
기호 자체를 정의한다.  |^^^^02d9|는 유니코드 코드포인트를 뜻하는
16진수이고 소문자로만 써야 한다.  숫자 대신 직접 문자를 입력해도 된다.
플레인텍에서도 쓸 수 있다.

\section{}
루비를 달 수 있다. ruby 패키지가 이미 존재하지만 \luatexko와 궁합이 잘
맞지 않아 새로 매크로를 제공한다.  플레인텍도 지원한다.
\begin{quote}
|\ruby{漢字}{한자}|\quad$\Rightarrow$\quad\ruby{漢字}{한자}
\end{quote}
이처럼 글자별로 따로 루비를 달 필요가 없다.  관련 설정은 다음처럼
한다.

①~|\rubyfont|: 루비를 식자할 폰트를 지시해 둔다.

②~|\def\rubysize{0.6}|: 루비 글자 크기를 본문 글자 크기에 대한 비율로
지정

③~|\def\rubysep{0.2ex}|: 루비와 본문 글자 사이의 간격을 지정

④~|\rubynooverlap|: 루비의 폭이 본문 글자의 폭보다 클 때 루비가 이웃
글자들 위로 삐져나가지 못하게 한다. 본문 글자의 흐름을 중시하여
|\rubyoverlap|을 기본값으로 하였으므로 이는 따로 선언할 필요가 없다.

\section{}
ulem 패키지가 \luatexko와 궁합이 잘 맞지 않아{\small (줄바꿈에 문제가 있음)}
명령을 따로 제공한다. 플레인텍에서도 쓸 수 있다.

\bigskip
\halign{\qquad#\hfil&\quad$\Rightarrow$\quad#\hfil\cr
|\uline{밑줄을 그을 수 있다}|&\uline{밑줄을 그을 수 있다}\cr
|\sout{취소선을 그을 수 있다}|&\sout{취소선을 그을 수 있다}\cr
|\uuline{밑줄을 두 줄 긋는다}|&\uuline{밑줄을 두 줄 긋는다}\cr
|\xout{빗금으로 취소할 수 있다}|&\xout{빗금으로 취소할 수 있다}\cr
|\uwave{물결표로 밑줄을 삼는다}|&\uwave{물결표로 밑줄을 삼는다}\cr
|\dashuline{대시로 밑줄을 삼는다}|&\dashuline{대시로 밑줄을 삼는다}\cr
|\dotuline{밑줄을 점선으로 긋는다}|&\dotuline{밑줄을 점선으로 긋는다}\cr
}
\bigskip

관련하여 다음 설정을 할 수 있다.

①~|\def\ulinedown{0.25em}|: 밑줄을 베이스라인 아래로 끌어내리는 정도

②~|\def\ulinewidth{0.04em}|: 밑줄의 굵기

\section{}
자동조사는 \kotex 과 동일하게 |\은| |\는| |\이| |\가| |\을| |\를|
|\와| |\과| |\로| |\으로| |\라| |\이라| 따위를 사용한다.  문장 중에서도
작동할 뿐만 아니라 플레인텍도 지원한다.

\section{}
항목 번호를 한국어 기호로 붙일 수 있다. \kotex과 동일하게 |\jaso|
|\gana| |\ojaso| |\ogana| |\pjaso| |\pgana| |\onum| |\pnum| |\oeng|
|\peng| |\hnum| |\Hnum| |\hroman| |\hRoman| |\hNum| |\hanjanum| 따위를
사용한다.

\section{}
hyperref 패키지는 |[unicode]| 옵션을 주어야 한다.

\section{}
|\luatexhangulnormalize=1|이라 지시하면 첫가끝 자모를 완성형 음절로,
|2|라면 완성형 음절을 첫가끝 자모로 인코딩 변환한다. |0|이 할당되면
인코딩 변환 기능이 꺼진다. \XeTeX의 |\XeTeXinputnormalization| 명령과
유사하나 오직 한글과 일부 한자에 대해서만 정규화가 작동하는 점에서
\XeTeX의 그것에 비해 기능이 한참 모자란다.

\section{}\label{sec:uhcencoding}
권장하지 않지만 불가피하게 입력 인코딩이 UHC {\small(CP949라고도 하며
EUC-KR을 포함한다)}인 파일을 처리할 때는 |\luatexuhcinputencoding=1|을
선언한다. |0|을 할당하면 다시 UTF-8 입력으로 간주한다.
\XeTeX의 |\XeTeXinputencoding| 명령과 유사하나 오직 한글만 처리할 수
있어 \XeTeX의 그것에 비해 기능이 한참 모자란다.

\section{}
마찬가지로 바람직하지는 않지만 불가피하게 파일 이름이 UHC로 인코딩되어
있다면 |\luatexuhcfilenames=1|을 선언한다. |0|을 할당하면 다시 UTF-8
이름으로 간주한다.  윈도 계열 운영체제에서만 문제될 것이다.

\begin{flushright}
  \fboxsep=-\fboxrule
  \fbox{\vbox to 1em{\hbox to 1em{\hss}\vss}}
\end{flushright}

\end{document}
